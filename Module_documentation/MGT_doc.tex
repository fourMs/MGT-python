\documentclass[9pt]{extarticle}
\usepackage[utf8]{inputenc}
\usepackage[T1]{fontenc}
\usepackage{float}
\usepackage{titling}
\usepackage{geometry}
\usepackage{authblk}
\usepackage{enumitem}

\title{Musical Gestures Toolbox\\ \large{Documentation}}
\author{Frida Furmyr \& Marcus Widmer}

\begin{document}
\maketitle
\section*{\texttt{motionfilter}}
\section*{\texttt{mg\_videoreader}}
\hspace{20pt}\texttt{mg\_videoreader(filename, starttime = 0, endtime = 0, skip = 0, contrast = 0, brightness = 0, crop = 'none')}

\section*{Class MgObject}
    \begin{verbatim}
    Initializes Musical Gestures data structure from a given parameter video file.

    Parameters:
    filename (str): Name of input parameter video file.
    method (str): Currently 'Diff' is the only implemented method. 
    filtertype (str): 'Regular', 'Binary', 'Blob' (see function motionfilter).
    thresh (float): a number in [0,1]. Eliminates pixel values less than given threshold.
    starttime (float): cut the video from this start time (min) to analyze what is relevant.
    endtime (float): cut the video at this end time (min) to analyze what is relevant.
    blur (str): 'Average' to apply a blurring filter, 'None' otherwise.
    skip (int): When proceeding to analyze next frame of video, this many frames are skipped.
    color (bool): True does the analysis in RGB, False in grayscale. 
    contrast (float): apply +/- 100 contrast to video
    brightness (float): apply +/- 100 brightness to video
    crop (str): 'none', 'manual', 'auto' to select cropping of relevant video frame size
    \end{verbatim}
    
\section*{mg\_videoreader}
    \begin{verbatim}
    mg_videoreader(filename, starttime = 0, endtime = 0, skip = 0, contrast = 0, brightness = 0, crop = 'none'):
    
    Reads in a video file, and by input parameters user decide if it: trims the length, skips frames, applies contrast/brightness adjustments and/or crops image width/height.
    
    filename (str): Name of input parameter video file.
    starttime (float): cut the video from this start time (min) to analyze what is relevant.
    endtime (float): cut the video at this end time (min) to analyze what is relevant.
    skip (int): When proceeding to analyze next frame of video, this many frames are skipped.
    contrast (float): apply +/- 100 contrast to video
    brightness (float): apply +/- 100 brightness to video
    crop (str): 'None', 'Auto' or 'Manual' to crop video.
    
    return:
    - vidcap: cv2 video capture of editevideo file
    - length, fps, width, height from vidcap
    - of: filename gets updated with whaprocedures it went through
    \end{verbatim}

\section*{motionvideo}    
    \begin{verbatim}
    motionvideo(self, method = 'Diff', filtertype = 'Regular', thresh = 0.001, blur = 'None', kernel_size = 5):
    
    Finds the difference in pixel value from one frame to the next in an input video, and saves the frames into a new video.
    Describes the motion in the recording.    
    Outputs a video called filename +'_motion.avi'.
    
    Parameters:
    kernel_size (int): Size of structuring element.
    method (str): Currently 'Diff' is the only implemented method. 
    filtertype (str): 'Regular', 'Binary', 'Blob'(see function motionfilter) 
    thresh (float): a number in [0,1]. Eliminate spixel values less than given threshold.
    blur (str): 'Average' to apply a blurring filter, 'None' otherwise.
    
    Returns:
    None
    \end{verbatim}

\section*{motionfilter}
    \begin{verbatim}
    motionfilter(motion_frame, filtertype,thresh,kernel_size)
    
    Apply a filter to a picture/videoframe
    
    motion_frame (array(uint8)): input motion image
    filtertype (str):
                ’Regular’, turns all values below thresh to 0,
                ’Binary’ turns all values below thresh to 0, above thresh to 1,
                ’Blob’ removes individual pixels with erosion method.
    thresh (float): for ’Regular’ and ’Binary’ option, thresh is a value of threshold [0,1];
    kernel_size(int): Size of structuring element
    
    return: filtered frame (array(uint8))
    \end{verbatim}

\section*{mg\_centroid}
    \begin{verbatim}
    mg_centroid(image, width, height):

    Computes the centroid of an image/frame.
    
    Parameters
    - image (uint8)
    - width/height of image
    
    Returns:
    - Centroid of motion: Where was the maximum change in pixel value
    - Quantity of motion: How large was the change in pixel value
    \end{verbatim}
    
\section*{constrainNumber}
    \begin{verbatim}
    constrainNumber(n, minn, maxn)
    Constrains number to having a value between minn and maxn
    
    Parameters:
    - n (number)
    - minn (lower limit n can be)
    - maxn (lower limit n can be)
    
    return:
    Constrained number
    \end{verbatim}


\section*{cropvideo}
    \begin{verbatim}
    cropvideo(fps,width,height, length, of, crop_movement = 'auto', motion_box_thresh = 0.1, motion_box_margin = 1)
	Crops the video.

	Parameters:
		- crop_movement: {'auto','manual'}
			'Auto' finds the bounding box that contains the total motion in the video.
			Motion threshold is given by motion_box_thresh.
			'manual' opens up a simple GUI that is used to crop the video manually 
			by looking at the first frame

		- motion_box_thresh: float
			Only meaningful is crop_movement = 'auto'. Takes floats between 0 and 1, 
			where 0 includes all the motion and 1 includes none
		
		- motion_box_margin: int
			Only meaningful is crop_movement = 'auto'. Add margin to the bounding box.
	Returns:
		- None
    \end{verbatim}

\section*{input_test}
    \begin{verbatim}
    input_test(filename,method,filtertype,thresh,starttime,endtime,blur,skip):
    """ Gives feedback to user if initialization from input went wrong. """
    Ex: raise InputError(msg)
    msg = 'Please specify a filter type as str: Regular or Binary'
    \end{verbatim}

\section*{motionhistory} 
    \begin{verbatim}
    motionhistory(self, history_length = 20, kernel_size = 5, method = 'Diff', filtertype = 'Regular', thresh = 0.001, blur = 'None'):

    Finds the difference in pixel value from one frame to the next in an input video, and saves the difference frame to a history tail. 
    The history frames are summed up and normalized, and added to the current difference frame to show the history of motion. 
    Outputs a video called filename + '_motionhistory.avi'.

    Parameters:
    history_length (int): How many frames will be saved to the history tail.
    kernel_size (int): Size of structuring element.
    method (str): Currently 'Diff' is the only implemented method. 
    filtertype (str): 'Regular', 'Binary', 'Blob' (see function motionfilter) 
	thresh (float): a number in [0,1]. Eliminates pixel values less than given threshold.
    blur (str): 'Average' to apply a blurring filter, 'None' otherwise.

    Returns:
    None
    \end{verbatim}

\section*{contrast\_brightness}
\begin{verbatim}
    contrast_brightness(of,vidcap,fps,width,height,contrast,brightness):
    Edit contrast and brightness of the video.
    
    of (str): filename without extension
    vidcap: cv2 capture of video file, with all frames ready to read with vidcap.read().
    fps, width, height are simply info about vidcap
    contrast (float): apply +/- 100 contrast to video
    brightness (float): apply +/- 100 brightness to video
    
    return: cv2 video capture of edited video file
\end{verbatim}

\section*{skip\_frames}
\begin{verbatim}
    skip_frames(of, vidcap, skip, fps, width, height)
    
    Frame skip, convenient for saving time/space in an analysis of less detail looking at big picture movement. Skips the given number of frames, making a compressed version of the input video file.
    
    of (str): filename without extension
    vidcap: cv2 capture of video file, with all frames ready to read with vidcap.read().
    fps, width, height are simply info about vidcap
    skip (int): When proceeding to analyze next frame of video, this many frames are skipped.
    
    return:
        cv2 video capture of edited video file
        length, fps, width, height from this video capture
\end{verbatim}
\section*{motionaverage}
    \begin{verbatim}
	Post-processing tool. Finds and saves an average image of entire video.
    
    Usage:
    from _motionaverage import motionaverage
    motionaverage('filename.avi')
    \end{verbatim}

\end{document}